
% tip to kill the wrong type
%!TEX program = xelatex
\documentclass[10pt,journal,compsoc]{IEEEtran}
\IEEEoverridecommandlockouts
% The preceding line is only needed to identify funding in the first footnote. If that is unneeded, please comment it out.
\usepackage{cite}
\usepackage{amsmath,amssymb,amsfonts}
\usepackage{algorithmic}
\usepackage{graphicx}
\usepackage{textcomp}
\usepackage{xcolor}
%\usepackage{}
\usepackage{algorithm}
\usepackage{xeCJK}
\def\BibTeX{{\rm B\kern-.05em{\sc i\kern-.025em b}\kern-.08em
    T\kern-.1667em\lower.7ex\hbox{E}\kern-.125emX}}
\usepackage{fancyhdr}
%\pagenumbering{roman}

%%
%kErrb3rt:玩家的游玩路线是这样的,从主世界开始,首先发展自己(在"apocolypse"的监视之下,并且由于出生之日apocolypse就把监管植入了玩家,所以玩家在哪个维度都受到信用点管控(我等下补充,你不用发散这个部分),然后玩家随着史诗,找到了死亡之冠,与溺尸王大战,最终打败溺尸王,开启地狱的进入权限(此时玩家击碎沉星导致的全世界地狱门开启,以及拿到了太平洋之风),到达地狱后,前往Tyrant暴君的宫殿,暴君依然存活,只是不爱搭理生命,但是他感受到了那股风,于是玩家和暴君大战,玩家击败Tyran后,得到了暴君之镐(一把武器)和「echo of vOid」虚空回响,一个罗盘(损坏的罗盘),当玩家回到主世界,立刻被apocolypse本体留下的监听部分发现,并且夺取了虚空回响,但是天启的残存部分没有料到,天启本体留下的这个监听其实是针对残存部分的,这个监听得到了力量后,自毁开启了末地传送门,但是只有「生命思想体」才能进入,天启的分身并没有思想,这部分对于玩家来说,只是打开了末地开启权限的作用,但是对于剧情还是有点重要的。
%玩家此时直接死亡,然后时间回到最初,玩家再次诞生,从新走过之前的路,不同的是溺尸王改为了soul of 桑恩,暴君改为了卢纳,显然这是Nova在阻止玩家见到他,因为第一轮回的时候Nova就混在玩家之中
\begin{document}
\section{普罗米修斯之火计划概况}
人类试图推翻主机和少数人压迫的计划

\section{人物}
\textbf{主要}:

多琳娜(Dolina):28岁,女生物学家,是第一大城市中心城的首席二十科学家之一,是和其伴侣泽塔一起努力进入伊甸园研究所的。

泽塔(Zeta):30岁,男计算机科学家,是多琳娜的伴侣,中心城最大研究所--伊甸园的首席计算机科学家。二人在外界合称”伊甸园枢纽二人--简称枢纽“

默瑟(Mercer):69岁,社会学家,服役过,会格斗,底层人,曾经是上一代伊甸园,而他的”配偶“是多琳娜(可能有点乱,后续会解释)。

洛米亚(Nyromia):17岁女黑客,不知从何而来,只有默瑟知道,她是伊甸园第二代湿件,生物计算机群--多米诺骨牌之一。只不过她被赋予了人格,默瑟一直在照顾她,因为她是多琳娜制造的湿件之一。

盖斯(G . Ith):38岁,矮小的工程师,矮小的原因是双腿变为了外骨骼,在大停电中发生以外,但是却得知大停电并非以外的人。


\section{背景}

主机迭代101的年代,人类受控于主机,下层人即无法死亡又无法拥有梦想,只能不断卖命,只为了不再往下掉,鬼知道下面还有几层。上层人类却是另外一种风景,他们认为主机是他们的朋友,只是工具。并且把下层称为卑贱的培养皿筛选器。主机在迭代80提出了伊甸园计划,意在打造最强的科学支持--实际上,伊甸园只是另一个计算的培养皿。而伊甸园研究所的其他人都从事着”表面的研究“。

伊甸园:主机提出的研究所计划,只有少数人直到其正真目的:为主机的计算提供波动:主机的计算无法打破概率的本质--量子化随机。而意图通过伊甸园计划筛选有潜质的人类情侣打破自己的限制,为了提高人员的利用率,所谓的大停电就是每代伊甸园换代的时候,多琳娜就是第18代伊甸园的枢纽二人之一:多琳娜和默瑟,但是由于主机的计算出不匹配而发动停电更换了更没潜质的默瑟,而默瑟通过洛米亚的神经潜入得到了伊甸园的真相。

普罗米修斯之火计划:给主机外挂量子模块使得主机的人情升高而解放全体人类的计划(人类历史记载版)

普罗米修斯之火计划(默瑟的实际计划):彻底摧毁主机,消灭伊甸园和中心城的毁灭计划,不惜可能死掉最下层的被主机控制的人类(几十万)

普罗米修斯之火计划(普罗米修斯小队内部提法):给主机外挂量子模块使得主机升级完成新飞跃的科学计划,并且可能具有解放大部分人从而实现巨大的对人类方的利好。

\section{故事大概}
\textbf{时间线顺序:}

首席科学家情侣:多琳娜和泽塔美好的生活在一次大停电中终止,泽塔失去一切,包括他的视觉,多琳娜却成为了新的首席科学家,但是不知到为何,新伊甸园只有一个人--当然伊甸园有几个人,应该有几个人世界上只有10个人以内清楚,这10个人包括默瑟。

插入:默瑟线(回忆):

默瑟一开始不清楚以前发生了什么,但是他遇到了那个女孩,或者说正在成长的机器,洛米亚。在一个雨夜,街上都是泥泞,只有默瑟和洛米亚站在街上,默瑟在抽烟,他不记得怎么出现在这里的,只知道是停电之后,他被炒鱿鱼了,正当他抽烟到下一根的时候,女孩凑了上来”你是默瑟吗“,”是的,但是。。。我没有犯罪。“女孩却说:”你一定不是这里生活的人,这里生活的人听见名字会直接死掉。“令默瑟感到差异,一个小女孩说出这么恐怖的话,如果是假的,那就是个小屁孩,如果是真的,那她为什么如此残忍以至于说出名字的时候还在微笑?”你是谁?为什么是我?“默瑟作为社会学家的神经发作,在确认还有两分钟清扫街道--清扫就是让外面的人全部脑死亡送去给主机做动力的提取。他把这个女孩带回了他的庇护所...(待续)

默瑟线(回忆)结果:默瑟知道自己的身份是伊甸园的弃子,他没有告诉任何人,只有洛米亚知道,而女孩是人肉黑客--多琳娜的杰作之一,肉体计算器湿件,代号多米诺骨牌的计算单元之一。

回到主时间线:默瑟警惕的发现这一次的普罗米修斯计划居然只有一个人,按理来说,伊甸园计划其实是人类精英配对计划,所以默瑟打听上次大停电,上层垃圾场捡到的东西有没有包括某个个精致的”人“。于是他找到了泽塔,并且叫洛米亚恢复了泽塔的记忆,但是他发现泽塔居然还记得多琳娜。可能这就是主机寻求的”解“就是他和多琳娜?他这样想着,一个爬行的人却找来了,他自称G.Ith,但是为了方便就叫他盖斯。盖斯说他是量子研究所的研究员,但是在停电的时候刚好处于夹板之间,他看到了同事瞬间被肢解,然后运输走了,再从管道里运输进了复制人,然后随着房屋还原系统启动,他的下半生卡在夹板中被且切断了,失去支持掉了下去,看见了骇人的一幕,盖斯的复制人,完美的有机体,站在原地在烧录他的记忆数据。盖斯只好用快速止血剂,加上外骨骼从夹板里逃到外面,坐上了垃圾车从上层到达了下层。默瑟听到后,知道是时候启动那个计划了,他把3个人召集到一起,提出了普罗米修斯计划:

\textbf{Mercer:"我们有计算机工程师,有量子研究所的专家,而我知道,我们都知道,主机能守旧并且操纵到今日就是因为缺少了关键的随机产生部件:量子火种。只有量子才能产生真正的随机数,成为量子计算机,伊甸园计划很明显,只是主机的一个是失败品罢了。我想给主机加上量子模块!"
}
,盖斯一瞬间就明白了默瑟的提议,他表达了堪忧”要是主机量子化后继续压迫和杀死人类怎么办?“,”同时研发量子化的破坏装置,要是主机量子化失败,直接彻底摧毁他,也利用随机状态!“盖斯听到后陷入沉默。一向活泼的洛米亚当然听不进去,早就睡着了。泽塔说,我们就4个人,该怎么做?默瑟稳操胜券的:”我们还缺一个上层人来帮助我们“,盖斯听了感到不可思议:”上层巴不得你们全部去最下层当电池!怎么可能帮你“,泽塔就想到了:”也许,可以试试,多琳娜?让洛米亚恢复她的记忆?“默瑟嘴角抽动:”那么上层的人物,没有机会给她恢复记忆,她会叫镇压员来的。“泽塔垂下头,蒙着黑布的眼睛似乎从一开始就无神的看着地板。"我会去试试,没有把握不会把她带过来。"


\end{document}
